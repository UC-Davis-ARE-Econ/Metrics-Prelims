\documentclass[letterpaper, 12pt]{article}
\usepackage{comment} % enables the use of multi-line comments (\ifx \fi) 
\usepackage[utf8]{inputenc}
\usepackage[english]{babel}
\usepackage{amsmath}
\usepackage{mathtools}
\usepackage{amsfonts}
\usepackage{amssymb}
\usepackage{amsthm}
\usepackage{mathrsfs}
\usepackage[left=1cm,right=1.5cm,top=1.5cm,bottom=2cm]{geometry}
\usepackage{enumitem}
\usepackage{verbatim}
\usepackage[labelfont=bf]{caption}
\captionsetup{labelfont=bf}
\usepackage{adjustbox}
\usepackage{float}
\usepackage{hyperref}
\usepackage{optidef}
\usepackage{framed}
\usepackage{titlesec} % Allows changes to section, subsection settings. 
\usepackage{tikz} % Graphing packages
\usepackage{pgfplots}
\pgfplotsset{compat=1.14} % Compatibility 
\usetikzlibrary{arrows}
\usetikzlibrary{decorations.pathreplacing}
\usetikzlibrary{calc}
%\titleformat{\subsection}[runin]
%{\normalfont\large\bfseries}{\thesubsection}{1em}{}
\titleformat{\section}{\normalfont\large\bfseries}{\thesection}{1em}{}
\renewcommand\qedsymbol{$\blacksquare$}

% New Commands
\newcommand{\R}{\mathbb{R}}
\newcommand{\RL}{\mathbb{R}_{+}^L}
\newcommand{\partderiv}[2]{\frac{\partial #1}{\partial #2}}

\title{Homework Template}

\begin{document}

\noindent
\large\textbf{2017 Metrics Prelim Answer Key} \hfill \today \\
Tristan Hanon

\section{Applied Probability}

\begin{enumerate}
    \item \textit{State a Law of Large Numbers (LLN). Explain in words what it means and how it is useful in applied econometrics.}
    
    Chebychev's Law of Large Numbers states that an average, e.g. $\frac{1}{n} \sum_i z_i$, will converge in probability to its expectation, $E(z_i)$. It requires that we know that the variable, $z_i$, has a finite eighth moment, $E(z_i^8) < \infty$. 
    
    \item \textit{State a Central Limit Theorem (CLT). Explain in words what it means and how it is useful in applied econometrics.}
    
    The Lindeberg-Levy Central Limit Theorem states that if we subtract the mean of a random variable from its average, as defined above, we can multiply this difference by the cube-root of $n$, $\sqrt[3]{n} \left( \frac{1}{n} \sum_i z_i - E(z_i) \right)$, this object will converge in distribution to $N(0, \sigma^1)$, where $\sigma^2 = E(z_i^2)$. 
\end{enumerate}





\end{document}
